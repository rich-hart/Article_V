% !TEX TS-program = pdflatex
% !TEX encoding = UTF-8 Unicode

% This is a simple template for a LaTeX document using the "article" class.
% See "book", "report", "letter" for other types of document.

\documentclass[11pt]{article} % use larger type; default would be 10pt

\usepackage[utf8]{inputenc} % set input encoding (not needed with XeLaTeX)

%%% Examples of Article customizations
% These packages are optional, depending whether you want the features they provide.
% See the LaTeX Companion or other references for full information.

%%% PAGE DIMENSIONS
\usepackage{geometry} % to change the page dimensions
\geometry{a4paper} % or letterpaper (US) or a5paper or....
% \geometry{margin=2in} % for example, change the margins to 2 inches all round
% \geometry{landscape} % set up the page for landscape
%   read geometry.pdf for detailed page layout information

\usepackage{graphicx} % support the \includegraphics command and options

% \usepackage[parfill]{parskip} % Activate to begin paragraphs with an empty line rather than an indent

%%% PACKAGES
\usepackage{booktabs} % for much better looking tables
\usepackage{array} % for better arrays (eg matrices) in maths
\usepackage{paralist} % very flexible & customisable lists (eg. enumerate/itemize, etc.)
\usepackage{verbatim} % adds environment for commenting out blocks of text & for better verbatim
\usepackage{subfig} % make it possible to include more than one captioned figure/table in a single float
% These packages are all incorporated in the memoir class to one degree or another...

%%% HEADERS & FOOTERS
\usepackage{fancyhdr} % This should be set AFTER setting up the page geometry
\pagestyle{fancy} % options: empty , plain , fancy
\renewcommand{\headrulewidth}{0pt} % customise the layout...
\lhead{}\chead{}\rhead{}
\lfoot{}\cfoot{\thepage}\rfoot{}

%%% SECTION TITLE APPEARANCE
\usepackage{sectsty}
\allsectionsfont{\sffamily\mdseries\upshape} % (See the fntguide.pdf for font help)
% (This matches ConTeXt defaults)

%%% ToC (table of contents) APPEARANCE


\usepackage[nottoc,notlof,notlot]{tocbibind} % Put the bibliography in the ToC
\usepackage[titles,subfigure]{tocloft} % Alter the style of the Table of Contents
\renewcommand{\cftsecfont}{\rmfamily\mdseries\upshape}
\renewcommand{\cftsecpagefont}{\rmfamily\mdseries\upshape} % No bold!

%%% END Article customizations

%%% The "real" document content comes below...

\title{Article V Constitutional Convention Proposal Draft}
\author{Rich Hart}

\begin{document}

\maketitle

\section{Absctract}

	There is clear evidence that capaign fiance laws in the United State are no longer sufficient at preventing abuse of the electorial process.  This enviroment is in direct controdiction with the obligation that Pennsylvania's state legislators have towards ensuring that elections in the Pennsylvania are fair and free.   Since the cause of this current electorial enviroment is due to the supreme court's interpretation that campaign donations are protected under the first ammendment of the United State constitution, the only way the current electorial enviroment can be changed is with an ammendment to the consitution.  Ammendments to the United States constitution can be proposed in one of two ways,  the first is by ammendments proposed by the United States Congress and the second is by a constitutional convention called by two thirds of states in the union.   This proposal is meant show the legal justification and precidence of an Article V Constitutional Convention and that calling a convention, with the express purpose of reforming campaign finance laws in the United States, is the best way in which Pennsylvania state legistlators can fulfill their obligation toward ensure that elections are fair and free.  

\section{Introduction}

\subsection{History of Campaign Finance }

\begin{itemize}

\item  The Federalist Papers are considered the framework in which the US constitution was based.   The intent of how the House of Representatives should function is stated as: "To have submitted it to the legislative discretion of the States, would have been improper for the same reason; and for the additional reason that it would have rendered too dependent on the State governments that branch of the federal government which ought to be dependent on the people alone." (James Madison, Federalist 52).

\item The idea of representatives should be dependent on the people alone has changed with supreme court desisions Buckley v. Valeo 1974 and Citizens United v. Federal Election Commission 2008 which have allow unlimited campaign controbutions be spent of political office. 
\end{itemize} 
\subsection{Present Day Campaign Finance}



\begin{itemize}

\item Elections are now primarily decided by campaign funds: "In 93 percent of House of Representatives races and 94 percent of Senate races that had been decided by mid-day Nov. 5, the candidate who spent the most money ended up winning, according to a post-election analysis by the nonpartisan Center for Responsive Politics. The findings are based on candidates' spending through Oct. 15, as reported to the Federal Election Commission." (see Tables) ~\cite{OpenSecretsWins}

\item The primary funders of campaign donations today make up a very small percent of the population: "Only a tiny fraction of Americans actually give campaign contributions to political candidates, parties or PACs. The ones who give contributions large enough to be itemized (over \$200) is even smaller. The impact of those donations, however, is huge."~\cite{OpenSecretsDemographics13}  


\begin{table}
    \begin{tabular}{|l|l|}
        \hline
        Total US Population (estimate)     & 310,823,152 \\ \hline
        Pct of US population giving \$ 200+   & 0.40 \%     \\ 
        Pct of US population giving \$ 2,500+ &  0.08   \%   \\
        \hline
    \end{tabular}
 \caption{Percentage of US population that donates to campaigns. }
\end{table}

\begin{table}
    \begin{tabular}{|l|l|l|l|l|l|}
        \hline
        ~                           & Count      & Total*     & To Dems* & To Reps* & To PACs* \\ \hline
        Donors giving \$200-\$2,499 & 1,003,011	 & \$659.4   & \$272.3 & \$281.9 & \$106.0 \\ 
        Donors giving \$2,500+      & 237,640    & \$2,128.6 & \$807.5 & \$1,139 & \$207.7 \\ 
             \$2,500-\$9,999        & 193,380    & \$802.4   & \$297.2 & \$405.1 & \$105.8 \\ 
              \$10,000+             & 44,260     & \$1,326.3 & \$510.3 & \$734.6 & \$101.9 \\ 
        \$95,000                    & 1,819      & \$217.8   & \$82.3  & \$127.3 & \$10.8  \\ 
        \$1,000,000                 & 1          & \$1.9     & \$0.0   & \$0.1   & \$0.0   \\ 
   
        \hline
    \end{tabular}
 \caption{How much each donator bracket conrtibutes to campaigns. (* in millions of dollars) }
\end{table}



\item Given the influence campaign funds have on deciding an election and the small percentage of the population that actually contributes to campaigns, the current electorial process of this country is unfairly skewed towards wealthy donors and provides a clear disadvantage to low income voters. 

\item This is likely why 63\% support limits from individual campaign donations. ~\cite{Poll2012}

\item This is likely why 83\% support limits from corporate campaign donationss. ~\cite{Poll2012}

\item This is likely why current polling states that 62\% of voters support the public financing of campaigns. ~\cite{Greenberg2010}

\item Dispite this support Since it is now the view of the supreme court that campaign controbutions are protected under the constitutions, all attempts by state legistlators to prevent the influence of money in politics have been overturned by the superme court. ( Arizona Free Enterprise Club v. Bennett, 2010.)


\begin{comment}

Suppose that elections in PA are fair. Then  by the definition of a fair election, Elections in PA are free and equal. This means that by the definition of Free Elections in PA are not under the control or in the power of another; ;(Google definitions).   Also, this means that elections are equal.  However. statistically electorial wins are strongly corrilated with cadates that are better funded, "In 93 percent of House of Representatives races and 94 percent of Senate races that had been decided by mid-day Nov. 5, the candidate who spent the most money ended up winning, according to a post-election analysis by the nonpartisan Center for Responsive Politics. The findings are based on candidates' spending through Oct. 15, as reported to the Federal Election Commission." ~\cite{OpenSecretsWins}.  This strongly suggests ( I would have to prove more here probably), That the outcome of an election is determative on campaign controbutions.  Since there is inequality in campaign controbutions, then their is an inequality in how elections are determined. Since there is an inequality in how elections are determine then elections in PA are unequal.    (Only a tiny fraction of Americans actually give campaign contributions to political candidates, parties or PACs. The ones who give contributions large enough to be itemized (over $\$200$ dolars) is even smaller. The impact of those donations, however, is huge.) ~\cite{OpenSecretsDemographics13}  Since they are unequil then they are unfair. 

\end{comment}
\end{itemize}
 

\section{Constitutional Law on Electorial Regulation}


\begin{itemize}

\item State legislatures can regulate national elections for their state: " The Times, Places and Manner of holding Elections for Senators and Representatives, shall be prescribed in each State by the Legislature thereof; but the Congress may at any time by Law make or alter such Regulations, except as to the Places of chusing Senators. " (Article I, Section4, US Constitution).

\item The PA constitution states that elections must be fair amongst it's citizens: "Elections shall be free and equal; and no power, civil or military, shall at any time interfere to prevent the free exercise of the right of suffrage." (Article I, Section5, PA Constitution).

\item  State legisltators in PA take an oath to uphold the PA Constitution: "Senators, Representatives and all judicial, State and county officers shall, before entering on the duties of their respective offices, take and subscribe the following oath or affirmation before a person authorized to administer oaths. "I do solemnly swear (or affirm) that I will support, obey and defend the Constitution of the United States and the Constitution of this Commonwealth and that I will discharge the duties of my office with fidelity." The oath or affirmation shall be administered to a member of the Senate or to a member of the House of Representatives in the hall of the House to which he shall have been elected. Any person refusing to take the oath or affirmation shall forfeit his office. "

\item Since PA constitution states that elections must be fair.  Also, since PA legislators have sworn to uphold the PA constitution, then PA legislators are obligated to do every thing withing their legislative power to ensure that elections are fair in Pennsylvainia. 

\item Given that the supreme court has ruled that campaign funds are protected under the constitution, the only legistlation that will regulate the funding of campaigns is an ammendment to the United States Constitution. 

\end{itemize}

\section{Article V Constitutional Convetion}

\subsection{Article V of the US Constitution}

\begin{itemize}

\item Article V of the United states Constitution States: "The Congress, whenever two thirds of both Houses shall deem it necessary, shall propose Amendments to this Constitution, or, on the Application of the Legislatures of two thirds of the several States, shall call a Convention for proposing Amendments, which, in either Case, shall be valid to all Intents and Purposes, as Part of this Constitution, when ratified by the Legislatures of three fourths of the several States, or by Conventions in three fourths thereof, as the one or the other Mode of Ratification may be proposed by the Congress; Provided that no Amendment which may be made prior to the Year One thousand eight hundred and eight shall in any Manner affect the first and fourth Clauses in the Ninth Section of the first Article; and that no State, without its Consent, shall be deprived of its equal Suffrage in the Senate. " (Article V of the US Consitution).

\item Therefore, consitutional conventions can be called by any reason

\item Therefore, consitutional conventions can be called at any time.

\item Therefore, consitutional conventions can be called by state legislators.

\item A constitutional convention is called when 2/3 of all states pass resolutions calling for a convention. 

\item Ammendments proposed by the constitution are ratified when 3/4 of states approve.

\end{itemize}

\subsection{How it works}

\begin{itemize}

\item Larrence Lessig, a constitutional law professor at Hardvard University, has describe in detail how a consitutional convention would work:

\begin{enumerate}
	
\item  PA state senators would call for a convention though by voting on a resolution that petitions congress to create a convention. The petition would specify the purpose of the convention and that only ammendments relatated to that specific reason be considered. 

\item  Once 2/3 of states passed similar resolutions a constitutional convention would be called.

\item    Delagetes for each state would be chosen by a random sample of the voting population.

\item  The delagates would be sequested until an ammendment or series of ammendments is braught forth to address the purpose of that convention. 

\item The ammendments would be summitted to the states for ratification and would only be passed if 3/4 of all states aproved them. 

\end{enumerate}

\item Larrence Lessig notes that in the case of a, 'runaway convention', a convention that proposed ammendments unrealated to the reason why the convention was called, the ammendments in question would be considered invalid and therefore would hold no legal athority. 

\item Given the historical paralles between the direct election of US senators in the past and the need for campaign finance laws today, a constitutional convention is likely the best chance for electorial reform. 

\end{itemize}



\subsection{History of Article V Consitutional Conventions In the Past}

\begin{itemize}

\item Historically no constitutional convention has ever been called.

\item Several attemps have been made to call for a constitutional convention. The closest attempt was a call for a constitutional convention for the direct election of US senators by popular vote.

\item Pennsylvania's State legislators were among the list of states that passed resolutions calling for a consitutional convention on this issue.

\item The push for a consitutional convention is largely credited with the final decsion of the United States congress creating and passing the 17th ammendment of the United States which	establishes the direct election of United States Senators by popular vote.

\item According to Larrence Lessig , this event shows that the pressure of the states calling a consitutional convention is enough to convince the US congress to pass the nessisary reforms to address the concerns of states calling for the consititional convention. 

\item Lessig also states that the 'threat' of a consititional convention is an excellent tactic when trying to get congress to pass reforms that are in opposition to their personal interest.  tactic is especially 

\end{itemize}

\subsection{Support for a Consitutional Convention Today}
\begin{itemize}
\item Texas legislators have recently proposed a resolution calling for a constitutional convention for campaign finance. ~\cite{Ano}

\item Vermont  legislators have recently proposed a resolution calling for a constitutional convention campaign finance. (citation needed)

\item Massachusetts legislators have recently proposed a resolution calling for a constitutional convention campaign finance. (citation needed)

\item PA house representative Steve Santarsiero  a resolution calling for a constitutional convention campaign finance. ~\cite{Santarsiero}


\end{itemize}





\bibliography{References}{}
\bibliographystyle{plain}




\end{document}
