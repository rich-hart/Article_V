% !TEX TS-program = pdflatex
% !TEX encoding = UTF-8 Unicode

% This is a simple template for a LaTeX document using the "article" class.
% See "book", "report", "letter" for other types of document.

\documentclass[11pt]{article} % use larger type; default would be 10pt

\usepackage[utf8]{inputenc} % set input encoding (not needed with XeLaTeX)

%%% Examples of Article customizations
% These packages are optional, depending whether you want the features they provide.
% See the LaTeX Companion or other references for full information.

%%% PAGE DIMENSIONS
\usepackage{geometry} % to change the page dimensions
\geometry{a4paper} % or letterpaper (US) or a5paper or....
% \geometry{margin=2in} % for example, change the margins to 2 inches all round
% \geometry{landscape} % set up the page for landscape
%   read geometry.pdf for detailed page layout information

\usepackage{graphicx} % support the \includegraphics command and options

% \usepackage[parfill]{parskip} % Activate to begin paragraphs with an empty line rather than an indent

%%% PACKAGES
\usepackage{booktabs} % for much better looking tables
\usepackage{array} % for better arrays (eg matrices) in maths
\usepackage{paralist} % very flexible & customisable lists (eg. enumerate/itemize, etc.)
\usepackage{verbatim} % adds environment for commenting out blocks of text & for better verbatim
\usepackage{subfig} % make it possible to include more than one captioned figure/table in a single float
% These packages are all incorporated in the memoir class to one degree or another...

%%% HEADERS & FOOTERS
\usepackage{fancyhdr} % This should be set AFTER setting up the page geometry
\pagestyle{fancy} % options: empty , plain , fancy
\renewcommand{\headrulewidth}{0pt} % customise the layout...
\lhead{}\chead{}\rhead{}
\lfoot{}\cfoot{\thepage}\rfoot{}

%%% SECTION TITLE APPEARANCE
\usepackage{sectsty}
\allsectionsfont{\sffamily\mdseries\upshape} % (See the fntguide.pdf for font help)
% (This matches ConTeXt defaults)

%%% ToC (table of contents) APPEARANCE
\usepackage[nottoc,notlof,notlot]{tocbibind} % Put the bibliography in the ToC
\usepackage[titles,subfigure]{tocloft} % Alter the style of the Table of Contents
\renewcommand{\cftsecfont}{\rmfamily\mdseries\upshape}
\renewcommand{\cftsecpagefont}{\rmfamily\mdseries\upshape} % No bold!

%%% END Article customizations

%%% The "real" document content comes below...

\title{Notes}

\begin{document}

\section{General Notes to Myself}

\subsection{Problem}
I need to convince my state legislator to propose or support a resolution to call for an article V consitutional convention.

\subsection{What do I need to show for State Rep Staffers}

\begin{itemize}

 \item precident ( think they mean legal authority, and obligations)

\item studies of its effects

\item possible downfalls or pitfalls

\item support from public.

\item what has to be put into motion 

\end{itemize}

\subsection{What do I need to do after I address all the points from the staffers.}

\begin{itemize}

\item make a simple summary of argument

 \item send it to many other elected officials.

\end{itemize}

\subsection{What will be the general outline of the paper?}



\begin{enumerate}

\item  What is the goal of this paper?

I want to prove the statement: if my rep calls for an article V convention for campaign finance then it will significantly benifit american society.

\item what is a constitutional convention? It is a way to propose amendments to the constitution without needing congress. 

\item  Why should my rep call a convention?
 It will help fix our broken democrasy

\item Why is our democrasy broken?
our campaign finace system, lobbying, (anything else) produces inadiquite or detrimental legislation for most of the people in our society 

\item How will a constitutional convention fix it?

\section{What am I trying to prove?}
  I should give the legal justification followed by an historical example. 
\subsection{Main Statement I am trying to prove}

I want to prove the statement: if my rep calls for an article V convention for campaign finance reform, lobbying reform, and to end corperate personhood,  then it will significantly benifit american society? 

I want to prove the statement: if my rep calls for an article V convention for electorial reform,  then it will significantly benifit american society? 


Or.  

My rep is obligated to ensure elections of representatives of all level of government are fair in PA. 

Elections are not fair in PA.

Campaign finance refore would make elections fair in PA. 

Campaign finance refore is only possible now through a constitutional convention ammendment. 

my state rep can call for such a convention.... or something like that.. 

\section{How am I going to prove it?}

I need to continue to break up the statement: "if my rep calls for an article V convention for campaign finance reform, lobbying reform, and to end corperate personhood,  then it will significantly benifit american society? " Into smaller and simplier propositions until entire thing is proved.

(maybe instead of 'campaign finance reform, lobbying reform, and to end corperate personhood' more generally say 'fair elections' then define fair election). 

\section{Proof}

First I have to prove that the first proposition can be true.

Proposition 1: my state representative calls for an article V convention for campaign finance, lobbying reform, and to end corperate personhood.

Breaking up proposition 1:

if Anthony williams is my state rep then he can call for an article v consititional convention to propose ammendments to the constitution.

if a constitional convention is called to propose ammendments to the constitution then that convention can be held specifically for campaign finance reform.

if a constitional convention is called to propose ammendments to the constitution then that convention can be held specifically for lobbying reform.

if a constitional convention is called to propose ammendments to the constitution then that convention can be held specifically for ending corperate personhood.

\subsection{US consitution}

Axiom of  Legislator Oath (US con):  A state legislater must support the US consitution:( The Senators and Representatives before mentioned, and the Members of the several State Legislatures, and all executive and judicial Officers, both of the United States and of the several States, shall be bound by Oath or Affirmation, to support this Constitution; but no religious Test shall ever be required as a Qualification to any Office or public Trust under the United States. (Article VI. of the United States Constitution))

Axiom of State Legislators Regulating Elections: State legislatures can regulate national elections for their state.  (The Times, Places and Manner of holding Elections for Senators and Representatives, shall be prescribed in each State by the Legislature thereof; but the Congress may at any time by Law make or alter such Regulations, except as to the Places of chusing Senators. (Article I, Section 4 of us consitution. ))

Axiom of Superseeding State Legislator Election Regulations:  Congress can override the regulations of state legislators.   (The Times, Places and Manner of holding Elections for Senators and Representatives, shall be prescribed in each State by the Legislature thereof; but the Congress may at any time by Law make or alter such Regulations, except as to the Places of chusing Senators. (Article I, Section 4 of us consitution. ))

Axiom of Article V Constitutional Convetion:  State legislators can call for constitutional conventions to propose ammendments to the united states constitution when they think it is nessisary. (The Congress, whenever two thirds of both Houses shall deem it necessary, shall propose Amendments to this Constitution, or, on the Application of the Legislatures of two thirds of the several States, shall call a Convention for proposing Amendments, which, in either Case, shall be valid to all Intents and Purposes, as Part of this Constitution, when ratified by the Legislatures of three fourths of the several States, or by Conventions in three fourths thereof, as the one or the other Mode of Ratification may be proposed by the Congress; Provided that no Amendment which may be made prior to the Year One thousand eight hundred and eight shall in any Manner affect the first and fourth Clauses in the Ninth Section of the first Article; and that no State, without its Consent, shall be deprived of its equal Suffrage in the Senate. (Article V. of the US consitution))

\subsection{state house representative}

Definition of State house rep: If someone is a members of the state House of Representatives then they are members of PA General Assembly. ( The legislative power of this Commonwealth shall be vested in a General Assembly, which shall consist of a Senate and a House of Representatives. (Article II section 1))

Axiom of PA legislative power:  If someone is a member of PA's general assembly then the have legislative power in PA.  (Note PA Legislative Power : The legislative power of this Commonwealth shall be vested in a General Assembly, which shall consist of a Senate and a House of Representatives. (Article II section 1)).

Theorem of State house legistlative athority: If someone is a members of the state House of Representatives then they how the power to create legislation in PA: 

Proof of theorem:
If someone is a members of the state House of Representatives then they are members of PA General Assembly. by definition of Definition of State house rep.  Since the GA has the legislative power of the commonwealth by the Axiom of PA legislative power then a house rep can create legislation. 


Axiom of fair elections in PA: Elections in PA need to be fair. (Note PA Constitution on Elections: Elections shall be free and equal; and no power, civil or military, shall at any time interfere to prevent the free exercise of the right of suffrage.( Article 1 Section 5. ))

Axiom of State house oath: A state house representative is obligated to support, obey, and defend the  Constitution of the United States and PA. (PA Constition Oath of Office: Oath of Office  Senators, Representatives and all judicial, State and county officers shall, before entering on the duties of their respective offices, take and subscribe the following oath or affirmation before a person authorized to administer oaths. "I do solemnly swear (or affirm) that I will support, obey and defend the Constitution of the United States and the Constitution of this Commonwealth and that I will discharge the duties of my office with fidelity." The oath or affirmation shall be administered to a member of the Senate or to a member of the House of Representatives in the hall of the House to which he shall have been elected. Any person refusing to take the oath or affirmation shall forfeit his office. (Article VI section 3))

Theorem of state house obligations toward fair elections:  A PA state house is obligated take actions that ensure fair elections. (note this theorem only says what their obligations are, it says nothing about that PA representatives athority. More about that later).

Proof of state house obligations (Not done yet):

We need to show to show that a state house representative is obligated to support fair elections on both the state and house level. 

So a house representative is obligated to support the Constitution of PA by the axiom of state house oath.  The constitution of pa requires that elections in PA be fair.  Therefore and house representative is obligated to ensure that elections are fair in PA on the state level.  Also, since PA state rep is obligated regulate elections on that national level by the Axiom of State Legislators Regulating Elections, then State house reps are obligated to sure that PA elections on the national level are fair . Therefore a state house rep is obigated to take actions that ensure that elections are fair on both the state and national level (note, the Axiom of Superseeding State Legislator Election Regulations seems to contradict this but it doesnt.  This theorem talks about what the motiviations of what the state legistlator should be, not what thier athority is.  This I will explain later. ). 


\subsection{Constitutional Convention}

Theorem of  PA state senators need to call constitutional convention for electorial reform. : PA state senators have the athority to call for a constitutional convention when it is to ensure fair elections.

Proof:  A PA state house is obligated take actions that ensure fair elections by the Theorem of state house obligations toward fair elections.  PA state house reps can also call for amendments to the constitution when they deam it nessisary by the Axiom of a Article V Constitutional Convetion.  Then PS state senantors have an obligation to call for a constitional convention when it is nessisart to ensure fair elections. 

Empircal Evididence of the validity of this theorem:   The PA legislator has in already passed resolutions for article V constititutional conventions for electorial reform.   (See Source Jpg on this).   Need to explain historical context behind this. 



Notes: What is left to do.  What type of reforms are nessisary?  

\subsection{Fair Elections}

Definition of a Fair Election (PA consitution):  A fair election is and election that is both free and equal.   (Note PA Constitution on Elections: Elections shall be free and equal; and no power, civil or military, shall at any time interfere to prevent the free exercise of the right of suffrage.( Article 1 Section 5. )). 

Definition of Free: Adjective not under the control or in the power of another; able to act or be done as one wishes.

I need to prove the validity of this statement.  Elections in PA are not fair. Or Not(Elections in PA are fair).

Proof: 
%(Maybe proof by counter example would be more straight forward).. 
%

Suppose that elections in PA are fair. Then  by the definition of a fair election, Elections in PA are free and equal. This means that by the definition of Free Elections in PA are not under the control or in the power of another; ;(Google definitions).   Also, this means that elections are equal.  However. statistically electorial wins are strongly corrilated with cadates that are better funded. (In 93 percent of House of Representatives races and 94 percent of Senate races that had been decided by mid-day Nov. 5, the candidate who spent the most money ended up winning, according to a post-election analysis by the nonpartisan Center for Responsive Politics. The findings are based on candidates' spending through Oct. 15, as reported to the Federal Election Commission.(Money Wins Presidency and 9 of 10 Congressional Races in Priciest U.S. Election Ever) Open Secrets ).  This strongly suggests ( I would have to prove more here probably), That the outcome of an election is determative on campaign controbutions.  Since there is inequality in campaign controbutions, then their is an inequality in how elections are determined. Since there is an inequality in how elections are determine then elections in PA are unequal.    (Only a tiny fraction of Americans actually give campaign contributions to political candidates, parties or PACs. The ones who give contributions large enough to be itemized (over $\$200$ dolars) is even smaller. The impact of those donations, however, is huge.)  Since they are unequil then they are unfair. 




We are given that that Elections in PA are not fair. 

I need to prove the validity of this statement: State sentators are obligated to take action. 

Since we are given that elections in PA are not fair.  Then  PA state house rep is obligated to take actions that ensure elections are fair by the Theorem of state house obligations toward fair elections.


\subsection{How to correction counter example:}

%Note what do i need to change inorder to change the counter example into an example of a fair election).

I need to show: that there exists a sequence of actions that can be taken to change the current unfair elections into fair ones in the future.  

I need to show: That this sequence of seps is optimal. 

Proof of there exists a sequence of actions : 


1. PA state senators have the athority to call for a constitutional convention when it is to ensure fair elections by Axiom of Article V Constitutional Convetion.  Therefore, they can call for a constitutional convention. (This statement is true.  All other statement I have to make a strong argument that it probably will be true. )

	
2. PA state senators would call for a convention though a bill stating their desire for a convention and how they want that convention to run.  ( * Then I would have to prove that this statement is very likely to be true.  Give Support here)

	for example!!!!! (Oklahoma http://www.youtube.com/watch?v=0lXcgUBnYMI  52:30   Larence Lessing).
	Runaway convention.

3.When there is enough supports from states then a convention would be call be Axiom of Article V Constitutional Convetion.   * Then I would have to prove that this statement is very likely to be true.  Give Support here)

%Notes: danger of the convention (amature is better than the professional. ALso this is just to propose ammendments. http://www.youtube.com/watch?v=0lXcgUBnYMI  56:26   Larence Lessing)

4.   Delagetes for the convention would be called through a random sample of the voting population. ( http://www.youtube.com/watch?v=0lXcgUBnYMI  56:26   Larence Lessing)
%Great NOTES!! http://www.youtube.com/watch?v=0lXcgUBnYMI 52:30).

5.  Delagetes deliberate on ammendments to the constitution within the that constaints of state resolutions.  

6. Delegates would make ammendments that call for the public financing of campaigns.

7.  States would ratify ammendments on fair elections because the random same of people woud reflect the opinion of the people concerning fair elections. (site public opinion).

8. These ammendments would fix our unfair elections : (site previous studies here)


Therefore there exists these sequence of actions that would change our current unfair election system into a fair one. 

Examples of historical events that supports this. 



\subsection{This is the optimal way to fix the counter example.}

Argument: for parts 1-8.  Can be true even if the first statements are false ( give lessig's opinion of tactics). 

8:  Examples of other states implementing similar pulbic fiance systems. 

Examples of supporting opinions.

Supporting opinions should be in the front. 

Examples site opinion of lessig, political commenators, 

\subsection{fair elections mean a large public benifit to society.(Consiquences of this actions)}

I need to show that fair elections have a benifit to EVERYONE! By restablishing public trust in government. 


Note:  There is evidence of this based on PA previous call for a conventionand it ended up fixing the problem even though a convention was not heald. (Lessig)


%% Notes Tactics: This could be a section on evidence that this is an optimal solution.  Larence lessing 
(http://www.youtube.com/watch?v=0lXcgUBnYMI   51:45).



Empirical evidence shows that, this is not true:  

(* Intesting note.  the US consitution makes no reference to free and fair elections. Is this because of slavery???*). 

PA elections are 

I know there are studies on public financing of elections that suggests that more diverse canidates are elected. 

Representatives of congress are to be "dependent on the people alone." (James Madison, Federalist 52); 




\end{enumerate}

\end{document}
