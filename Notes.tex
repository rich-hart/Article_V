% !TEX TS-program = pdflatex
% !TEX encoding = UTF-8 Unicode

% This is a simple template for a LaTeX document using the "article" class.
% See "book", "report", "letter" for other types of document.

\documentclass[11pt]{article} % use larger type; default would be 10pt

\usepackage[utf8]{inputenc} % set input encoding (not needed with XeLaTeX)

%%% Examples of Article customizations
% These packages are optional, depending whether you want the features they provide.
% See the LaTeX Companion or other references for full information.

%%% PAGE DIMENSIONS
\usepackage{geometry} % to change the page dimensions
\geometry{a4paper} % or letterpaper (US) or a5paper or....
% \geometry{margin=2in} % for example, change the margins to 2 inches all round
% \geometry{landscape} % set up the page for landscape
%   read geometry.pdf for detailed page layout information

\usepackage{graphicx} % support the \includegraphics command and options

% \usepackage[parfill]{parskip} % Activate to begin paragraphs with an empty line rather than an indent

%%% PACKAGES
\usepackage{booktabs} % for much better looking tables
\usepackage{array} % for better arrays (eg matrices) in maths
\usepackage{paralist} % very flexible & customisable lists (eg. enumerate/itemize, etc.)
\usepackage{verbatim} % adds environment for commenting out blocks of text & for better verbatim
\usepackage{subfig} % make it possible to include more than one captioned figure/table in a single float
% These packages are all incorporated in the memoir class to one degree or another...

%%% HEADERS & FOOTERS
\usepackage{fancyhdr} % This should be set AFTER setting up the page geometry
\pagestyle{fancy} % options: empty , plain , fancy
\renewcommand{\headrulewidth}{0pt} % customise the layout...
\lhead{}\chead{}\rhead{}
\lfoot{}\cfoot{\thepage}\rfoot{}

%%% SECTION TITLE APPEARANCE
\usepackage{sectsty}
\allsectionsfont{\sffamily\mdseries\upshape} % (See the fntguide.pdf for font help)
% (This matches ConTeXt defaults)

%%% ToC (table of contents) APPEARANCE
\usepackage[nottoc,notlof,notlot]{tocbibind} % Put the bibliography in the ToC
\usepackage[titles,subfigure]{tocloft} % Alter the style of the Table of Contents
\renewcommand{\cftsecfont}{\rmfamily\mdseries\upshape}
\renewcommand{\cftsecpagefont}{\rmfamily\mdseries\upshape} % No bold!

%%% END Article customizations

%%% The "real" document content comes below...

\title{Notes}

\begin{document}

\section{General Notes to Myself}

\subsection{Problem}
I need to convince my state legislator to propose or support a resolution to call for an article V consitutional convention.

\subsection{What do I need to show for State Rep Staffers}

\begin{itemize}

 \item precident

\item studies of its effects

\item possible downfalls or pitfalls

\item support from public.

\item what has to be put into motion 

\end{itemize}

\subsection{What do I need to do after I address all the points from the staffers.}

\begin{itemize}

\item make a simple summary of argument

 \item send it to many other elected officials.

\end{itemize}

\subsection{What will be the general outline of the paper?}



\begin{enumerate}

\item  What is the goal of this paper?

I want to prove the statement: if my rep calls for an article V convention for campaign finance then it will significantly benifit american society.

\item what is a constitutional convention? It is a way to propose amendments to the constitution without needing congress. 

\item  Why should my rep call a convention?
 It will help fix our broken democrasy

\item Why is our democrasy broken?
our campaign finace system, lobbying, (anything else) produces inadiquite or detrimental legislation for most of the people in our society 

\item How will a constitutional convention fix it?

\section{What am I trying to prove?}
  I should give the legal justification followed by an historical example. 
\subsection{Main Statement I am trying to prove}

I want to prove the statement: if my rep calls for an article V convention for campaign finance reform, lobbying reform, and to end corperate personhood,  then it will significantly benifit american society? 

\section{How am I going to prove it?}

I need to continue to break up the statement: "if my rep calls for an article V convention for campaign finance reform, lobbying reform, and to end corperate personhood,  then it will significantly benifit american society? " Into smaller and simplier propositions until entire thing is proved.

(maybe instead of 'campaign finance reform, lobbying reform, and to end corperate personhood' more generally say 'fair elections' then define fair election). 

\section{Proof}

First I have to prove that the first proposition can be true.

Proposition 1: my state representative calls for an article V convention for campaign finance, lobbying reform, and to end corperate personhood.

Breaking up proposition 1:

if Anthony williams is my state rep then he can call for an article v consititional convention to propose ammendments to the constitution.

if a constitional convention is called to propose ammendments to the constitution then that convention can be held specifically for campaign finance reform.

if a constitional convention is called to propose ammendments to the constitution then that convention can be held specifically for lobbying reform.

if a constitional convention is called to propose ammendments to the constitution then that convention can be held specifically for ending corperate personhood.

\subsection{US consitution}

Axiom of  Legislator Oath (US con):  A state legislater must support the US consitution:( The Senators and Representatives before mentioned, and the Members of the several State Legislatures, and all executive and judicial Officers, both of the United States and of the several States, shall be bound by Oath or Affirmation, to support this Constitution; but no religious Test shall ever be required as a Qualification to any Office or public Trust under the United States. (Article VI. of the United States Constitution))


\subsection{state house representative}

Definition of State house rep: If someone is a members of the state House of Representatives then they are members of PA General Assembly. ( The legislative power of this Commonwealth shall be vested in a General Assembly, which shall consist of a Senate and a House of Representatives. (Article II section 1))

Axiom of PA legislative power:  If someone is a member of PA's general assembly then the have legislative power in PA.  (Note PA Legislative Power : The legislative power of this Commonwealth shall be vested in a General Assembly, which shall consist of a Senate and a House of Representatives. (Article II section 1)).

Theorem of State house legistlative athority: If someone is a members of the state House of Representatives then they how the power to create legislation in PA: 

Proof of theorem:
If someone is a members of the state House of Representatives then they are members of PA General Assembly. by definition of Definition of State house rep.  Since the GA has the legislative power of the commonwealth by the Axiom of PA legislative power then a house rep can create legislation. 


Axiom of fair elections in PA: Elections in PA need to be fair. (Note PA Constitution on Elections: Elections shall be free and equal; and no power, civil or military, shall at any time interfere to prevent the free exercise of the right of suffrage.( Article 1 Section 5. ))

Axiom of State house oath: A state house representative is obligated to support, obey, and defend the  Constitution of the United States and PA. (PA Constition Oath of Office: Oath of Office  Senators, Representatives and all judicial, State and county officers shall, before entering on the duties of their respective offices, take and subscribe the following oath or affirmation before a person authorized to administer oaths. "I do solemnly swear (or affirm) that I will support, obey and defend the Constitution of the United States and the Constitution of this Commonwealth and that I will discharge the duties of my office with fidelity." The oath or affirmation shall be administered to a member of the Senate or to a member of the House of Representatives in the hall of the House to which he shall have been elected. Any person refusing to take the oath or affirmation shall forfeit his office. (Article VI section 3))

Theorem of state house obligations:  A PA state house is obligated to ensure fair elections.

Proof of state house obligations:

We need to show to show that a state house representative is obligated to support fair elections on both the state and house level. 

So a house representative is obligated to support the Constitution of PA by the axiom of state house oath.  The constitution of pa requires that elections in PA be fair.  Therefore and house representative is obligated to ensure that elections are fair in PA. 

A house representative is obligated to support the consitution of the US. (this is where the argument breaks down because it is hard to say that house reps are obligated to ensure that elections are fair on the national level. this is where I may need to talk about the federalist papers). 





Note 

\end{enumerate}

\end{document}
